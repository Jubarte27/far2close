\documentclass[a4paper,12pt,oneside]{article}
\usepackage[utf8]{inputenc}
\usepackage[T1]{fontenc}
\usepackage{times}
\usepackage[margin=3cm]{geometry}
%\usepackage[margin=2cm,left=3cm]{geometry}
\usepackage[english,brazilian]{babel} %french
\usepackage{colortbl}
\usepackage{fancyhdr}
\usepackage{lastpage}
\usepackage{multicol}
\usepackage{multirow}
\usepackage{array}
\usepackage{booktabs}
\usepackage{pdfpages}
%\usepackage{subcaption}
\usepackage[hyperindex=true]{hyperref}
\usepackage{xspace}
\usepackage{qsymbols}
\usepackage{soul}
\usepackage[portuges]{varioref}
%\usepackage[brazilian]{minitoc}
%\usepackage[rubberchapters,clearempty,pagestyles]{titlesec}
\usepackage{tabulary}
\usepackage{tabularx}

%\usepackage[round,sort,colon]{natbib}
\usepackage[numbers,sort]{natbib}
\let\cite\citep

%%%%%%%%%%%%%%%%%%%%%%%%%%%%%%%%%%%%%%%%%%%%%%%%%%%%%%%%%%%%%%%%%%%%%%%%%%%%%%%
% Título definido aqui
\newcommand{\titulo}{Aprimoramento dos métodos de busca de dados em espaços métricos}
\newcommand{\engtitle}{Improvement of search methods in metric spaces}
%%%%%%%%%%%%%%%%%%%%%%%%%%%%%%%%%%%%%%%%%%%%%%%%%%%%%%%%%%%%%%%%%%%%%%%%%%%%%%%

%%%%%%%%%%%%%%%%%%%%%%%%%%%%%%%%%%%%%%%%%%%%%%%%%%%%%%%%%%%%%%%%%%%%%%%%%%%%%%%
% Edital de submissão
\newcommand{\edital}{nome do edital}
%%%%%%%%%%%%%%%%%%%%%%%%%%%%%%%%%%%%%%%%%%%%%%%%%%%%%%%%%%%%%%%%%%%%%%%%%%%%%%%

%%%%%%%%%%%%%%%%%%%%%%%%%%%%%%%%%%%%%%%%%%%%%%%%%%%%%%%%%%%%%%%%%%%%%%%%%%%%%%%

% linha do header
\renewcommand{\headrulewidth}{0.0pt}
% linha do footer
\renewcommand{\footrulewidth}{0.0pt}

\pagestyle{fancy}                       % Sets fancy header and footer
\fancyhf{}
%\fancyhead[C]{\footnotesize \edital \hfill João V. F. Lima}
\fancyhead[LE,RO]{\footnotesize\thepage}
%\fancyhead[RE]{\footnotesize \nouppercase{\leftmark}}
%\fancyhead[LO]{\footnotesize \nouppercase{\rightmark}}
%\fancyfoot[C]{\footnotesize \edital}

\fancypagestyle{plain}{
  \fancyhead{}
  \fancyfoot{}
  \renewcommand{\headrulewidth}{0pt}
  \renewcommand{\footrulewidth}{0pt}
}

%\def\cleardoublepage{\clearpage\if@twoside \ifodd\c@page\else%
%  \hbox{}%
%  \thispagestyle{empty}%              % Empty header styles
%  \newpage%
%  \if@twocolumn\hbox{}\newpage\fi\fi\fi}


%\renewcommand*{\backref}[1]{}
%\renewcommand*{\backrefalt}[4]{%
%\ifcase #1 %
%(Not cited.)%
%\or
%(Cited on page~#2.)%
%\else
%(Cited on pages~#2.)%
%\fi}
%\renewcommand*{\backrefsep}{, }
%\renewcommand*{\backreftwosep}{ and~}
%\renewcommand*{\backreflastsep}{ and~}

%\let\cite\citep

\newcommand{\memoinclui}[3]{
  \newpage
%  \subsection{#1}
%  \includepdf[pages=-,frame=true,scale=.86,pagecommand={},noautoscale=false,offset={7mm -6mm}]{#2}
  \includepdf[pages=1,frame=true,scale=.86,pagecommand=\subsection{#1}\label{#2},noautoscale=false,offset={7mm -6mm}]{#3}
  \includepdf[pages=2-,frame=true,scale=.86,pagecommand={},noautoscale=false,offset={7mm -6mm}]{#3}
}

\newcommand{\memoincluipg}[3]{
  \newpage
  \includepdf[pages=1,frame=true,scale=.86,pagecommand=\subsection{#1}\label{#2},noautoscale=false,offset={7mm -6mm}]{#3}
}

\usepackage{color}
\definecolor{linkcol}{rgb}{0,0,0.4} 
\definecolor{citecol}{rgb}{0.5,0,0} 

% Change this to change the informations included in the pdf file
\hypersetup
{
  bookmarksopen=true,
  pdftitle="\titulo",
  pdfauthor="Matheus Machado Cezar", 
  pdfsubject="\titulo", 
  pdftoolbar=false, 
  pdfmenubar=true,
  pdfhighlight=/O,
  pdfpagemode=UseNone,
  pdfpagelayout=SinglePage,
  pdffitwindow=true,
  colorlinks=false,
%  colorlinks=true,
%  linkcolor=linkcol,
%  citecolor=citecol,
%  urlcolor=linkcol
}

\graphicspath{{.}{figuras/}}

%\renewcommand\bibname{Referências bibliográficas}

%\date{\today}
%\date{Dezembro de 2014}

\title{\titulo}

\author{Matheus Machado Cezar}

% minitoc 
%\setlength{\stcindent}{5pt}
%\renewcommand{\stcSSfont}{\small}

\begin{document}

\begin{titlepage}

\pagestyle{empty}
\begin{center}
\MakeUppercase{Universidade Federal de Santa Maria}\\
\MakeUppercase{Centro de Tecnologia}\\
%\MakeUppercase{Programa de Pós-Graduação em Informática}\\
\vspace*{\fill}
%\noindent \Large{Edital \edital} \\
%\vspace{2cm}
\vspace*{\fill}
\noindent \Large{Projeto de Pesquisa Científica} \\
\noindent \Large{Relatório Inicial} \\
\vspace*{\fill}
\noindent\Huge{\bf \titulo} \\
%\vspace{2cm}
\vspace*{\fill}
\noindent \Large{Matheus Machado Cezar} \\
%\noindent \Large{Laboratório de Sistemas de Computação} \\
%\vfill
\vspace*{\fill}
\normalsize{Agosto de 2023}
\end{center}

\end{titlepage}

\pagestyle{fancy}

%\setcounter{secnumdepth}{2}
\tableofcontents            % 
\newpage

\def\manycore{\textit{manycore}\xspace}
\def\multicore{\textit{multicore}\xspace}
\def\smp{\textit{Symmetric multiprocessor}\xspace}
\def\cores{\textit{cores}\xspace}
\def\multithread{\textit{multithread}\xspace}
\def\tbb{Intel\copyright~TBB}
\def\threads{\textit{threads}\xspace}
\def\thread{\textit{thread}\xspace}
\def\openmp{\textit{Open Multi-Processing}\xspace}
\def\crs{\textit{Cluster-aware Random Stealing}\xspace}
\def\clusters{\textit{clusters}\xspace}
\def\cluster{\textit{cluster}\xspace}
\def\mpi{\textit{Message-Passing Interface}\xspace}
\def\ampi{\textit{Adaptive MPI}\xspace}
\def\tbbex{\textit{Threading Building Blocks}\xspace}
\def\xkaapi{XKaapi\xspace}
\def\ufsm{Universidade Federal de Santa Maria\xspace}
\def\xeonphi{Intel Xeon Phi\xspace}

\newcommand{\xcol}{\cellcolor[gray]{0.4}}

% LSC
\def\nprofdoutores{4\xspace}
\def\ndoutor{1\xspace}
\def\nmestrandos{5\xspace}



%Projeto de pesquisa, contendo: título; dados de identificação; objetivos;
%metodologia; caracterização e relevância do tema, orçamento (Anexo I),
%cronograma de execução e plano de atividades do bolsista (atendendo aos termos
%e objetivos deste edital) e resultados esperados e impactos ambientais,
%econômicos, sociais e/ou de inovação; 

% 1 identificacao da proposta
%%%%%%%%%%%%%%%%%%%%%%%%%%%%%%%%%%%%%%%%%%%%%%%%%%%%%%%%%%%%%%%%%%%%%%%%%%%%%%%
\section{Identificação do projeto}
%%%%%%%%%%%%%%%%%%%%%%%%%%%%%%%%%%%%%%%%%%%%%%%%%%%%%%%%%%%%%%%%%%%%%%%%%%%%%%%

Título: {\bf \titulo}\\

\begin{otherlanguage}{english}
\noindent
Título em inglês: {\bf \engtitle}\\
\end{otherlanguage}

\noindent
Coordenador: Sérgio Luís Sardi Mergen\\

\noindent
Instituição executora: \ufsm \\

%\noindent
%Grupo de pesquisa/CNPq: Laboratório de Sistemas de Computação (LSC/UFSM)\\

\noindent
%Edital: --- \edital \\

\noindent
Início previsto: agosto de 2022 \\

\noindent
Duração: 24 meses \\

\noindent
Pesquisadores:
\begin{itemize}
\item Sérgio Luís Sardi Mergen, UFSM, Coordenador
\item Matheus Machado Cezar, Colaborador
\end{itemize}

%%%%%%%%%%%%%%%%%%%%%%%%%%%%%%%%%%%%%%%%%%%%%%%%%%%%%%%%%%%%%%%%%%%%%%%%%%%%%%%
\subsection{Resumo}

Espaços métricos são estruturas matemáticas que generalizam a noção de distância entre elementos de um conjunto. Eles são essenciais para medir a proximidade ou similaridade entre elementos em diversos contextos. A busca por dados em espaços métricos é um problema importante e está presente em diversas aplicações, como busca por texto e reconhecimento de padrões. O projeto de pesquisa visa desenvolver um método eficiente de armazenamento e busca para dados nesse contexto, que seja capaz de lidar com diferentes métricas, distribuições e quantidade de dados.

%%%%%%%%%%%%%%%%%%%%%%%%%%%%%%%%%%%%%%%%%%%%%%%%%%%%%%%%%%%%%%%%%%%%%%%%%%%%%%%
% 2 qualificação do problema
\section{Introdução}
%%%%%%%%%%%%%%%%%%%%%%%%%%%%%%%%%%%%%%%%%%%%%%%%%%%%%%%%%%%%%%%%%%%%%%%%%%%%%%%

O espaço métrico é um conceito matemático que generaliza o espaço euclidiano. Sua definição é baseada em um conjunto de pontos onde a distância entre eles é determinada por uma função chamada métrica. Essa função atribui a cada par de pontos um número real não negativo. A métrica mais comum é a distância euclidiana, utilizada para medir a distância entre dois pontos em um plano cartesiano, com a fórmula $\sqrt{(p_1-q_1)^2 + (p_2-q_2)^2}$.

Existem outras métricas que podem ser usadas, como a distância de Levenshtein, que mede a distância entre duas strings, a distância de Hamming, utilizada para strings de mesmo tamanho, e a distância de Jaccard, que mede a distância entre dois conjuntos.

Quanto às estruturas de armazenamento e busca em espaços métricos, várias opções estão disponíveis, sendo as árvores as mais comuns, como a Vantage Point Tree (VPTree) e a Cover Tree. Além disso, há estruturas baseadas em grafos, como a Graph of Nearest Neighbors (GNN) e a Graph of Metric Nearest Neighbors (GMNN).

Diversos tipos de consultas podem ser realizados em espaços métricos, incluindo a consulta por k-vizinhos mais próximos, e a consulta por vizinhos em um raio.


\section{Justificativa}

A busca por dados em espaços métricos é um desafio recorrente em diversas aplicações. Um exemplo prático dessa situação é a busca por texto, onde a distância de Levenshtein é útil para encontrar palavras similares à palavra buscada ou mesmo para correção de erros de digitação.

O cerne deste projeto de pesquisa está focado na busca por dados em espaços métricos. Nossa meta é identificar uma estrutura de dados que demonstre alto desempenho nessa tarefa. Para avaliar o desempenho da estrutura de dados selecionada, serão realizados testes abrangentes, considerando métricas como tempo de resposta e custo adicional de armazenamento.


%%%%%%%%%%%%%%%%%%%%%%%%%%%%%%%%%%%%%%%%%%%%%%%%%%%%%%%%%%%%%%%%%%%%%%%%%%%%%%%
% 4 metodologia a ser empregada
\section{Metodologia}
\label{sec:metodologia}
%%%%%%%%%%%%%%%%%%%%%%%%%%%%%%%%%%%%%%%%%%%%%%%%%%%%%%%%%%%%%%%%%%%%%%%%%%%%%%%

Para alcançar os objetivos delineados para o projeto, uma sequência de etapas foi elaborada. A descrição de cada uma delas virá a seguir.


\begin{description}

\item [Estudar o estado da arte:] Será realizado um levantamento de artigos e documentos que abordam os métodos de armazenamento e busca conhecidos em espaços métricos. A análise detalhada desses materiais permitirá compreender as diferenças entre os métodos existentes e como eles se complementam.

\item [Implementar estrutura de dados:] Com base no estudo do estado da arte, será escolhida uma estrutura de dados adequada para a busca de dados em espaços métricos. Essa estrutura será implementada em Java, utilizando as melhores práticas de programação.

\item [Implementar método de armazenamento:] A estrutura de dados desenvolvida anteriormente será adaptada para possibilitar o armazenamento em memória externa. Novamente, o método escolhido será embasado no estudo do estado da arte e a implementação será feita em Java.

\item [Reduzir tempo de resposta:] Será realizada uma análise minuciosa do método implementado, identificando pontos passíveis de otimização. Com base nessa análise, serão propostas melhorias no método para reduzir o tempo de resposta das buscas.

\item [Realizar testes:] Para avaliar o desempenho do método implementado, será criado um framework de testes específico para testar árvores de busca em espaço métrico. Esses testes serão conduzidos tanto com dados sintéticos quanto com dados reais, fornecendo uma avaliação abrangente da estrutura de dados desenvolvida.

\item [Documentação:] Todos os resultados obtidos ao longo do projeto serão devidamente documentados em arquivos em LaTeX. Em momentos oportunos, esses documentos poderão ser utilizados para a escrita de artigos científicos relatando as contribuições alcançadas no projeto.

\end{description}



%%%%%%%%%%%%%%%%%%%%%%%%%%%%%%%%%%%%%%%%%%%%%%%%%%%%%%%%%%%%%%%%%%%%%%%%%%%%%%%
% 7 cronograma de atividades
\section{Cronograma de atividades}
%%%%%%%%%%%%%%%%%%%%%%%%%%%%%%%%%%%%%%%%%%%%%%%%%%%%%%%%%%%%%%%%%%%%%%%%%%%%%%%

A Tabela~\vref{tab:cronograma} resume o cronograma de atividades baseado nas
atividades propostas na seção~\ref{sec:metodologia}, organizado por trimestres,
em um total de {\bf 24 meses}.
%

\begin{table}[!htb]
\centering
\begin{tabular}{|l|c|c|c|c|c|c|c|c|}
\cline{2-9}
\multicolumn{1}{c}{} &  \multicolumn{8}{|c|}{{\bf Trimestre}} \\
\hline
\multicolumn{1}{|c|}{{\bf Atividade}}   & 1$^o$ & 2$^o$ & 3$^o$ & 4$^o$ & 5$^o$ & 6$^o$ & 7$^o$ & 8$^o$ \\
\hline
Estudar estado da arte.                 & \xcol & \xcol &       &       &       &       &       &       \\
\hline
Implementar estrutura                   &       & \xcol & \xcol & \xcol &       &       &       &       \\
\hline
Implementar método de armazenamento     &       &       &       & \xcol & \xcol & \xcol &       &       \\
\hline
Reduzir tempo de resposta.              &       &       & \xcol & \xcol & \xcol & \xcol & \xcol &       \\
\hline
Realizar testes.                        &       &       & \xcol & \xcol & \xcol & \xcol & \xcol &       \\
\hline
Redação de resumos e artigos.           &       &       &       & \xcol & \xcol & \xcol & \xcol & \xcol \\
\hline
\end{tabular}
\caption{Cronograma das principais etapas no desenvolvimento do projeto de pesquisa.}
\label{tab:cronograma}
\end{table}

\section{Objetivos}

O objetivo deste projeto de pesquisa é desenvolver um método de armazenamento e busca altamente eficiente para dados em espaços métricos. Esse método será especialmente projetado para apresentar um excelente desempenho em consultas por k-vizinhos mais próximos e consultas por vizinhos em um raio.

Uma característica fundamental do método é a sua capacidade de lidar com diferentes tipos de métricas, permitindo que seja aplicado em uma ampla variedade de cenários. Além disso, o método será projetado para lidar com dados de diferentes distribuições e tamanhos, garantindo que seu desempenho seja robusto e escalável.

\section{Resultados Esperados}

Espera-se que os resultados obtidos ao longo do projeto possibilitem a publicação de artigos científicos em periódicos e conferências de alto impacto. Esses artigos serão focados na descrição e avaliação do método de armazenamento e busca desenvolvido, bem como no framework de testes criado para avaliar árvores de busca em espaço métrico. Além disso, espera-se a disponibilização de um conjunto de dados sintéticos e reais que serão utilizados para testar e validar o método.

\bibliographystyle{plainnat}
\addcontentsline{toc}{section}{Referências}
\bibliography{projeto}

\end{document}

